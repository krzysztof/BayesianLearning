\documentclass{article}
\usepackage{geometry}
\usepackage[utf8]{inputenc}
\usepackage{placeins}
\usepackage{hyperref}
\usepackage{amssymb,amsmath}
\usepackage[usenames,dvipsnames]{xcolor}
\usepackage{polski}
\usepackage{hyperref}
\usepackage{graphicx}
\usepackage{longtable}
%\title{TYTUL \\ TYTUL2 \\ }
%\author{\textsc{Krzysztof Nowak}}
%\author{\textsc{Krzysztof Nowak} \\ \textsc{Anna Warpechowska}}

\begin{document}
%\maketitle
%\begin{tabular}{lr}
%Prowadzący zajęcia: dr inż. Magdalena Topczewska
%\end{tabular}

\setlength\parindent{0pt} % Removes all indentation from paragraphs

\renewcommand{\labelenumi}{\alph{enumi}.} % Make numbering in the enumerate environment by letter rather than number (e.g. section 6)

%----------------------------------------------------------------------------------------
%	Przyklady
%----------------------------------------------------------------------------------------

\section{Calculating LEAK}

As it was shown in previous chapter, calculating LEAK naively may lead to problems with certain prior probabilities that favor high probability of parent being in state true.
\begin{longtable}{|l||p{2cm}|}
	\hline
	state & p \\\hline\hline
	true & 0.9 \\\hline
	false & 0.1 \\\hline
\end{longtable}

Such circumstances tend to produce data with few records modeling leak parameter directly.
Since leak parameter can lead to errors in calculating parameters further.

\begin{verbatim}
1 | 0 0 0 |   10 | 0.0010%
2 | 1 0 0 |   86 | 0.0086%
3 | 0 1 0 |   90 | 0.0090%
4 | 0 0 1 |  108 | 0.0108%
5 | 1 0 1 |  801 | 0.0801%
6 | 1 1 0 |  813 | 0.0813%
7 | 0 1 1 |  844 | 0.0844%
8 | 1 1 1 | 7248 | 0.7248%
\end{verbatim}

As we can see above, eliciting leak directly requires us to rely on 10 records (out of 10.000).
This carries considerable error in further calculations.

We can calculate leak parameter, by expressing leak as another variable that's always present.

\begin{verbatim}
1 | 0 0 0 (1) |   10 | 0.0010%
2 | 1 0 0 (1) |   86 | 0.0086%
3 | 0 1 0 (1) |   90 | 0.0090%
(...)
\end{verbatim}

We can solve the leak parameter by choosing a subset of the data file that satisfies the following condition:

\begin{description}
	\item[Condition for leak solvability] \hfill \\
        There exist a linear combination of vectors that produces a vector $[0 0 .. 0 (1)]$.
\end{description}

Since we don't care about solving parameters other than leak yet, the determinant of such matrix does not have to meet Rouche-Capelli conditions for unique solution.
If we choose vectors 6, 7 and 8 for our equation set, we can easily solve it for leak.
\begin{verbatim}
6 | 1 1 0 (1)|  813 | 0.0813%
7 | 0 1 1 (1)|  844 | 0.0844%
8 | 1 1 1 (1)| 7248 | 0.7248%
\end{verbatim}

\begin{equation}
    \begin{pmatrix}
        1 & 1 & 0 & 1 \\
        0 & 1 & 1 & 1 \\
        1 & 1 & 1 & 1
    \end{pmatrix} 
\end{equation}

\begin{description}
	\item[Opis boldem] \hfill \\
	Opis ponizej
\end{description}

\begin{longtable}{|l|l||p{2cm}|}
	\hline
	A & B & C\\\hline\hline
	0 & 1 & 4\\\hline
	a & b & e\\\hline
\end{longtable}

%\begin{figure}[h!]

%  \centering
%  \includegraphics[width=\textwidth]{knn.png}
%  \caption{OPIS}
%  \label{fig:knn}
%\end{figure}
\end{document}

\begin{description}
	\item[Opis boldem] \hfill \\
	Opis ponizej
\end{description}

\begin{longtable}{|l|l||p{2cm}|}
	\hline
	A & B & C\\\hline\hline
	0 & 1 & 4\\\hline
	a & b & e\\\hline
\end{longtable}

%\begin{figure}[h!]

%  \centering
%  \includegraphics[width=\textwidth]{knn.png}
%  \caption{OPIS}
%  \label{fig:knn}
%\end{figure}
\end{document}
