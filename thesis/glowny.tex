\documentclass[oneside,12pt]{wipb}

\katedra{Informatyki Teoretycznej}
\typpracy{ %magisterska
           %inżynierska
	   licencjacka
         }
\temat{Algebra z dzieleniem}
\autor{Marcin Koźniewski}
\promotor{dr hab. Piotr Grzeszczuk}
\indeks{75804}
\studia{stacjonarne
        %niestacjonarne
       }
\rokakademicki{2011/2012}
\profil{%magisterskie jednolite
        %magisterskie uzupełniające
        studia I stopnia
        %studia II stopnia
}
\kierunekstudiow{%informatyka
                 matematyka
                }
\specjalnosc{%Inżynieria Oprogramowania
             %Inżynieria Komputerowa
             %Systemy Oprogramowania
             %Metody infnformatyczne w~banknkowości i~finansach
             %Ochrona systemów informatycznych
	     Matematyczne metody informatyki
            }
\zakres{1. Konstrukcja kwaternionów Hamiltona\newline
\mbox{2. Konstrukcja algebr kwaternionów nad różnymi ciałami}\newline
\mbox{3. Własności algebr kwaternionów}\newline
\mbox{4. Twierdzenie Wedderburna o skończonych pierścieniach z
dzieleniem}\newline
\mbox{5. Zastosowania algebr kwaternionów}
%1. zakres I \newline 2. zakres II \newline 3. zakres III
}

\hypersetup{ %wpisy w pdf info
pdfauthor={Marcin Koźniewski},
pdftitle={Algebra z dzieleniem.
Praca licencjacka},
pdfsubject={},
pdfkeywords={Praca dyplomowa},
pdfpagemode=UseNone,
linkcolor=black,
citecolor=black
} 
\usepackage{geometry}
\usepackage[utf8]{inputenc}
\usepackage{placeins}
\usepackage{hyperref}
\usepackage{amssymb,amsmath}
\usepackage[usenames,dvipsnames]{xcolor}
\usepackage{polski}
\usepackage{hyperref}
\usepackage{graphicx}
\usepackage{longtable}
\usepackage{wasysym}
\makeatletter
\renewcommand*\env@matrix[1][*\c@MaxMatrixCols c]{%
  \hskip -\arraycolsep
  \let\@ifnextchar\new@ifnextchar
  \array{#1}}
\makeatother




\begin{document}


\newtheorem{twierdzenie}{Twierdzenie}[chapter]
\newtheorem{lemat}[twierdzenie]{Lemat}
\newtheorem{definicja}{Definicja}[chapter]
\newtheorem{przyklad}{Przykład}[chapter]
\newtheorem{wlasnosci}{Własności}[chapter]

\maketitle
\setcounter{page}{3}
\tableofcontents
\thispagestyle{empty}
%\setcounter{page}{0}
\pagestyle{plain}
	
\section{Solving parameters using Gauss-Jordan elimination}

Standard approach to solving determined sets of equations requires just enough constrains to ensure uniqueness of the solution, without it being overdetermined (which can lead to contradictions).
In our case, the set of available information is greater than required number of equations that satisfy above conditions.
Out of $2^n$ possible binary vectors representing joint probability of events, we are forced to pick $n$ that introduce the fewest error in further calculations.
This approach can yield good results when done properly, but the question of which equations to choose remains unanswered.
Preserving linear independence of vectors invoke additional complexity to the rules by which we decide the final set of equations.

Using Gauss-Jordan elimination, we can avoid this problem entirely, since it allows us to work with both overdetermined and underdetermined systems of equations.
The order of equations is also taken into account, so in cases of contradictions, certain combinations are preferred to others.
General rules of Gauss elimination apply, but are slightly modified to fit our product-equations (rather than linear equations).
The outcome of our procedure for Gauss-Jordan elimination is a reduced row echelon form.

\subsection{Example of Gauss-Jordan elimination}
Let us work with this simple example written in a standard matrix form $A \cdot X = b$.
\begin{equation}
    \begin{bmatrix}
        1 & 1 & 1 & 0 \\
        1 & 1 & 0 & 0 \\
        0 & 0 & 0 & 1 \\
        0 & 0 & 1 & 0
    \end{bmatrix} \cdot
    \begin{bmatrix}
        x_1 \\ x_2 \\ x_3 \\ x_4
    \end{bmatrix} = 
    \begin{bmatrix}
        b_1 \\ b_2 \\ b_3 \\ b_4
    \end{bmatrix}
\end{equation}

We will use symbols for vector \textbf{$b$}, so we can keep track of operations with constants.
Turning that into augmented matrix $[A|b]$ yields

\begin{equation}
\begin{bmatrix}[cccc|c]
    1 & 1 & 1 & 0 & b_1 \\ 
    1 & 1 & 0 & 0 & b_2 \\ 
    0 & 0 & 0 & 1 & b_3 \\ 
    0 & 0 & 1 & 0 & b_4
\end{bmatrix}
\end{equation}

Notice that this equation set is both contradictory ($x_3$ can be calculated in two ways), and underdetermined (There is no way to distinguish between $x_1$ and $x_2$ parameters).
We will now perform Gauss-Jordan elimination steps in order to show that certain properties we care about (such as preserving preference of equations determined by their order) apply.

Since we do not focus on any particular column, so we will try to solve the whole equation.
We will distinguish pivot elements with colors red (currently selected pivot element) and blue (previous pivot elements).

\begin{enumerate}
\item We choose the our first pivot element (in red), and use it to zero-out remaining coefficients in first column.
\begin{equation}
\begin{bmatrix}[cccc|c]
    \textcolor{red}{1} & 1 & 1 & 0 & b_1 \\ 
    1 & 1 & 0 & 0 & b_2 \\ 
    0 & 0 & 0 & 1 & b_3 \\ 
    0 & 0 & 1 & 0 & b_4
\end{bmatrix}
\begin{matrix} \\ r_2 = r_2 - r_1 \\ \\ \\ \end{matrix} \sim
\begin{bmatrix}[cccc|c]
    \textcolor{blue}{1} & 1 & 1 & 0 & b_1 \\ 
    0 & 0 & -1 & 0 & b_2 - b_1 \\ 
    0 & 0 & 0 & 1 & b_3 \\ 
    0 & 0 & 1 & 0 & b_4
\end{bmatrix}
\end{equation}
\item We select the second pivot element - note that no two pivot elements can share the same row.
The first non-zero element that satisfies this condition is coefficient of $x_3$ in second row.
Choosing the first element from the top guarantees that the order of preference of equations in taken into account.
\begin{equation}
\begin{bmatrix}[cccc|c]
    \textcolor{blue}{1} & 1 & 1 & 0 & b_1 \\ 
    0 & 0 & \textcolor{red}{-1} & 0 & b_2 - b_1 \\ 
    0 & 0 & 0 & 1 & b_3 \\ 
    0 & 0 & 1 & 0 & b_4
\end{bmatrix}
\begin{matrix} \\ r_2 = r_2 \cdot (-1) \\ \\ \\ \end{matrix} \sim
\begin{bmatrix}[cccc|c]
    \textcolor{blue}{1} & 1 & 1 & 0 & b_1 \\ 
    0 & 0 & \textcolor{red}{1} & 0 & b_1 - b_2 \\ 
    0 & 0 & 0 & 1 & b_3 \\ 
    0 & 0 & 1 & 0 & b_4
\end{bmatrix}
\end{equation}
\begin{equation}
\begin{bmatrix}[cccc|c]
    \textcolor{blue}{1} & 1 & 1 & 0 & b_1 \\ 
    0 & 0 & \textcolor{red}{1} & 0 & b_1 - b_2 \\ 
    0 & 0 & 0 & 1 & b_3 \\ 
    0 & 0 & 1 & 0 & b_4
\end{bmatrix}
\begin{matrix} r_1 = r_1 - r_2\\ \\ \\ r_4 = r_4 - r_2\end{matrix} \sim
\begin{bmatrix}[cccc|c]
    \textcolor{blue}{1} & 1 & 0 & 0 & b_1 - (b_1 - b_2) \\ 
    0 & 0 & \textcolor{blue}{1} & 0 & b_1 - b_2 \\ 
    0 & 0 & 0 & 1 & b_3 \\ 
    0 & 0 & 0 & 0 & b_4 - (b_1 - b_2)
\end{bmatrix}
\end{equation}

\item The last pivot element is going to be coefficient at $x_4$ in fourth row.
Since the remaining coefficients are all zeros in fourth column, no changes are made.
We can simplify the values in new vector $b$.
Notice that fourth row is a zero-vector - we can eliminate that from the equation set.
\begin{equation}
\begin{bmatrix}[cccc|c]
    \textcolor{blue}{1} & 1 & 0 & 0 & b_1 - (b_1 - b_2) \\ 
    0 & 0 & \textcolor{blue}{1} & 0 & b_1 - b_2 \\ 
    0 & 0 & 0 & \textcolor{red}{1} & b_3 \\ 
    0 & 0 & 0 & 0 & b_4 - (b_1 - b_2)
\end{bmatrix}
\sim
\begin{bmatrix}[cccc|c]
    \textcolor{blue}{1} & 1 & 0 & 0 & b_2 \\ 
    0 & 0 & \textcolor{blue}{1} & 0 & b_1 - b_2 \\ 
    0 & 0 & 0 & \textcolor{blue}{1} & b_3 \\ 
\end{bmatrix}
\end{equation}
\end{enumerate}

Let's compare our end-result with the initial matrix:
\begin{equation}
\begin{bmatrix}[cccc|c]
    1 & 1 & 1 & 0 & b_1 \\ 
    1 & 1 & 0 & 0 & b_2 \\ 
    0 & 0 & 0 & 1 & b_3 \\ 
    0 & 0 & 1 & 0 & b_4
\end{bmatrix}
\end{equation}

As we can see, $x_3$ was calculated using first and second row ($b_1 - b_2$).
Let's see what happens after we move third row on the top position, indicating that our preferred ordering of equation changed. (We expect now to calculate $x_3$ solely by first row).

\begin{enumerate}
\item We choose the our first pivot element, and use it to zero-out remaining coefficients in first column.
\begin{equation}
\begin{bmatrix}[cccc|c]
    0 & 0 & 1 & 0 & b_1 \\
    \textcolor{red}{1} & 1 & 1 & 0 & b_2 \\ 
    1 & 1 & 0 & 0 & b_3 \\ 
    0 & 0 & 0 & 1 & b_4 \\ 
\end{bmatrix}
\begin{matrix} \\ \\ r_3 = r_3 - r_2 \\ \\ \end{matrix} \sim
\begin{bmatrix}[cccc|c]
    0 & 0 & 1 & 0 & b_1 \\
    \textcolor{blue}{1} & 1 & 1 & 0 & b_2 \\ 
    0 & 0 & -1 & 0 & b_3 - b_2 \\ 
    0 & 0 & 0 & 1 & b_1 \\ 
\end{bmatrix}
\end{equation}
\item
Again, no candidate for pivot element in second column, coefficient at $x_3$ in first row is the next pivot element
\begin{equation}
\begin{bmatrix}[cccc|c]
    0 & 0 & \textcolor{red}{1} & 0 & b_1 \\
    \textcolor{blue}{1} & 1 & 1 & 0 & b_2 \\ 
    0 & 0 & -1 & 0 & b_3 - b_2 \\ 
    0 & 0 & 0 & 1 & b_1 \\ 
\end{bmatrix}
\begin{matrix} \\ r_2 = r_2 - r_1 \\ r_3 = r_3 + r_1 \\ \\ \end{matrix} \sim
\begin{bmatrix}[cccc|c]
    0 & 0 & \textcolor{blue}{1} & 0 & b_1 \\ 
    \textcolor{blue}{1} & 1 & 0 & 0 & b_2-b_1 \\ 
    0 & 0 & 0 & 0 & b_3 - b_2 + b_1 \\ 
    0 & 0 & 0 & 1 & b_4
\end{bmatrix}
\end{equation}.
\item Last item fit for a pivot element is a coefficient at $x_4$ in fourth row.
After getting rid of zero vectors, we achieve the following reduced row echelon form matrix
\begin{equation}
\begin{bmatrix}[cccc|c]
    0 & 0 & \textcolor{blue}{1} & 0 & b_1 \\ 
    \textcolor{blue}{1} & 1 & 0 & 0 & b_2-b_1 \\ 
    0 & 0 & 0 & 0 & b_3 - b_2 + b_1 \\ 
    0 & 0 & 0 & \textcolor{red}{1} & b_4
\end{bmatrix}
\sim
\begin{bmatrix}[cccc|c]
    0 & 0 & \textcolor{blue}{1} & 0 & b_1 \\ 
    \textcolor{blue}{1} & 1 & 0 & 0 & b_2-b_1 \\ 
    0 & 0 & 0 & \textcolor{blue}{1} & b_4
\end{bmatrix}
\end{equation}.
\end{enumerate}

As expected, $x_3$ was calculated using the most preferred set of equations, as dictated by their order.
What this method does not take into account is the relative weight (or preference) of each row.
As of yet, all we could rely on was simple ordering of equations, without using the information about quantity or frequency of each type of equation in our learning set.
If our method was to provide that feature to us as well, we could talk about very complete and solid solution that can be expected to perform optimally.

\subsection{Properties of zero vectors}
In this section we will describe how solving equation set without any particular ordering does not DEPRIVE[FIXME] us from reproducing other ways to calculate given parameter.
As we saw previously, different order of equation leads to different outcomes for certain parameters.
This variety came from contradictions in equation set.
Zero vectors that emerge during Gauss-Jordan elimination contain information about other ways to calculate given value.
Instead of removing them in the process, we can store them, and utilize them later.
Let's see how previous example holds to that theory.

\begin{equation}
\begin{bmatrix}[cccc|c]
    1 & 1 & 1 & 0 & b_1 \\ 
    1 & 1 & 0 & 0 & b_2 \\ 
    0 & 0 & 0 & 1 & b_3 \\ 
    0 & 0 & 1 & 0 & b_4
\end{bmatrix}
\sim
\begin{bmatrix}[cccc|c]
    1 & 1 & 0 & 0 & b_1 - (b_1 - b_2) \\ 
    0 & 0 & 1 & 0 & b_1 - b_2 \\ 
    0 & 0 & 0 & 1 & b_3 \\ 
    \textcolor{blue}{0} & \textcolor{blue}{0} & \textcolor{blue}{0} & \textcolor{blue}{0} &\textcolor{blue}{b_4 - (b_1 - b_2)}
\end{bmatrix}
\sim
\begin{bmatrix}[cccc|c]
    \textcolor{blue}{1} & 1 & 0 & 0 & b_2 \\ 
    \textcolor{red}{0} & \textcolor{red}{0} & \textcolor{red}{1} & \textcolor{red}{0} & \textcolor{red}{b_1 - b_2} \\ 
    0 & 0 & 0 & \textcolor{blue}{1} & b_3 \\ 
\end{bmatrix}
\end{equation}

This particular order of equation lead to $x_3$ being calculated from two top-most equations in a set.
If we add our final solution for $x_3$ (vector in red), to the zero vector we ought to remove in a penultimate step of our algorithm (vector in blue), we obtain previously abandoned solution:
\begin{equation}
\begin{bmatrix}[cccc|c]
    0 & 0 & 1 & 0 & b_1-b_2
\end{bmatrix}
+
\begin{bmatrix}[cccc|c]
    0 & 0 & 0 & 0 & b_4 - (b_1 - b_2) \\ 
\end{bmatrix}
=
\begin{bmatrix}[cccc|c]
    0 & 0 & 1 & 0 & b_4
\end{bmatrix}
\end{equation}

Using the same method we can start from the solution obtained after rearranging the order of equations in initial matrix.
\begin{equation}
\begin{bmatrix}[cccc|c]
    0 & 0 & 1 & 0 & b_1 \\
    1 & 1 & 1 & 0 & b_2 \\ 
    1 & 1 & 0 & 0 & b_3 \\ 
    0 & 0 & 0 & 1 & b_4 \\ 
\end{bmatrix}
\sim
\begin{bmatrix}[cccc|c]
    0 & 0 & 1 & 0 & b_1 \\ 
    1 & 1 & 0 & 0 & b_2-b_1 \\ 
    \textcolor{blue}{0} & \textcolor{blue}{0} & \textcolor{blue}{0} & \textcolor{blue}{0} &\textcolor{blue}{b_3 - b_2 + b_1} \\ 
    0 & 0 & 0 & 1 & b_4
\end{bmatrix}
\sim
\begin{bmatrix}[cccc|c]
    \textcolor{red}{0} & \textcolor{red}{0} & \textcolor{red}{1} & \textcolor{red}{0} &\textcolor{red}{b_1} \\ 
    1 & 1 & 0 & 0 & b_2-b_1 \\ 
    0 & 0 & 0 & 1 & b_4
\end{bmatrix}
\end{equation}

Linear combination of two vectors again yields a different solution

\begin{equation}
\begin{bmatrix}[cccc|c]
    0 & 0 & 1 & 0 & b_1
\end{bmatrix}
+
(-1)\cdot
\begin{bmatrix}[cccc|c]
    0 & 0 & 0 & 0 & b_3 - b_2 + b_1 \\ 
\end{bmatrix}
=
\begin{bmatrix}[cccc|c]
    0 & 0 & 1 & 0 & b_2-b_3
\end{bmatrix}
\end{equation}

Above property is crucial in showing that although the method itself does not take relative frequency, or weight between the equations in a set, it does provide us with a collection of solutions for given parameter.
This way we can iterate over the set of possible solutions and choose the one that minimizes the relative error, or take a weighted average as our solution.

Let's treat fourth column that responds to quantity of each equation in a dataset.
Whole equation set adds up to 2500 records.
We can calculate a frequency for each of the probabilities, and treat them as weights.

\begin{equation}
\begin{matrix}[cccc|c|c|c]
     &  &  &  & \mbox{probability} & \mbox{quantity} & \mbox{frequency Fq($b_i$)} \\
    \hline
    0 & 0 & 1 & 0 & b_1 & 980 & 0.392 \\
    1 & 1 & 1 & 0 & b_2 & 760 & 0.304 \\ 
    1 & 1 & 0 & 0 & b_3 & 440 & 0.176 \\ 
    0 & 0 & 0 & 1 & b_4 & 320 & 0.128 \\ 
\end{matrix}
\end{equation}

Using this data we can come up with few heuristics for calculating final value of $x_3$ or compare different solutions.
Let's propose a fitness function for a solution:

\begin{equation}
    F(s) = \frac{\displaystyle\prod_{b_{i} \in s} Fq(b_i)}{ \displaystyle\sum_{b_i \in s}1},
\end{equation},
where $b_i \in s$ is true when $b_i$ is taken into account (adding or subtracting) in given solution.

\begin{equation}
\begin{matrix}[cccc|c|c]
     &  &  &  & \mbox{probability} & \mbox{fitness function F(s)} \\
    \hline
    [0 & 0 & 1 & 0]_1 & b_1 & 0.392 \\
    [0 & 0 & 1 & 0]_2 & b_2 - b_3 & \frac{0.304 \cdot 0.176}{2} = 0.24 \\ 
\end{matrix}
\end{equation}

At this point we can pick a solution with a higher fitness value, or take weighted average of each solution as our final answer:

\begin{equation}
\begin{matrix}[cccc|c]
     &  &  &  & \mbox{probability}\\
    \hline
    [0 & 0 & 1 & 0] & \frac{0.392 \cdot b_1 + 0.24 \cdot (b_2 - b_3)}{0.392+0.24}\\
\end{matrix}
\end{equation}


\section{Different approaches to solving parameters}
Since our method enables us to propose different solutions for each parameter (in cases of overdetermined systems), we can propose few means of choosing the final value for given parameter.
These can rely on choosing the best solution out of all that are possible, or combining several best choices together.
\begin{enumerate}
    \item \textbf{Choose the best candidate for a solution out of $k$ best guesses.}

        We have to remember that in cases of ambiguity for each parameter, there are infinitely many linear combinations of first solution and the zero vectors.
        If we would like to pick the best solution, we have to restrict our search space to the subset of $k$ best candidates.
        In order to compare candidates for a solution, we can propose a fitness function that tries to estimate the error associated with each solution.
        %Ideally, our method would yield the best solution as the initial first solution in reduced row echelon form, which can be achieved if we include the idea behind given fitness function in Gauss-Jordan elimination.
    \item \textbf{Combine $k$ possible candidates for solution into a final answer}
        
        Once we elicit the set of $k$ possible candidates, we can use weighted average to combine them into final result.
        This will not automatically improve the result, but will average-out the error, which in most cases may result good solutions.
        This may be especially useful in case where it's difficult to propose a fitness function that describes the error accurately.
\end{enumerate}
Since both approaches rely on similar input (fitness function, initial set of $k$ candidates) we'll try to describe it first.

\begin{description}
    \item[Candidates for the solution]

    As we had shown previously, outcome of Gauss--Jordan elimination is just one of the possible solutions in overdetermined systems of equations.
\end{description}

\begin{description}
    \item[Candidates for fitness function] \hfill \\
    \begin{enumerate}
        \item \textbf{Relative frequency of each combination in a data file}
        Intuitively it is clear that the error is inversely proportional to the frequency of given combination within the data file.
        First good guess for a fitness function could be the frequency with which given equation appears in a data file.
    \end{enumerate}
\end{description}

\subsection{Examples}
Let us append some quality measures to each equation in the dataset.
The whole equation set adds up to 2500 records.
We can calculate a frequency for every vector, and treat it as a weight.

\begin{equation}
\begin{matrix}[cccc|c|c|c]
     &  &  &  & \mbox{probability} & \mbox{quantity} & \mbox{frequency Fq($b_i$)} \\
    \hline
    0 & 0 & 1 & 0 & b_1 & 980 & 0.392 \\
    1 & 1 & 1 & 0 & b_2 & 760 & 0.304 \\ 
    1 & 1 & 0 & 0 & b_3 & 440 & 0.176 \\ 
    0 & 0 & 0 & 1 & b_4 & 320 & 0.128 \\ 
    \hline
     &  &  &  &  & 2500 & \\ 
\end{matrix}
\end{equation}

Using this data we can propose few heuristics for calculating final value of $x_3$ or compare different solutions.
Let us propose a fitness function for a solution:

\begin{equation}
    F(s) = \frac{\displaystyle\prod_{b_{i} \in s} Fq(b_i)}{ \displaystyle\sum_{b_i \in s}1},
\end{equation},
where $b_i \in s$ is true when $b_i$ is taken into account (adding or subtracting) in given solution.

\begin{equation}
\begin{matrix}[cccc|c|c]
     &  &  &  & \mbox{probability} & \mbox{fitness function F(s)} \\
    \hline
    [0 & 0 & 1 & 0]_1 & b_1 & 0.392 \\
    [0 & 0 & 1 & 0]_2 & b_2 - b_3 & \frac{0.304 \cdot 0.176}{2} = 0.24 \\ 
\end{matrix}
\end{equation}

At this point we can pick a solution with a higher fitness value, or take weighted average of each solution as our final answer:

\begin{equation}
\begin{matrix}[cccc|c]
     &  &  &  & \mbox{probability}\\
    \hline
    [0 & 0 & 1 & 0] & \frac{0.392 \cdot b_1 + 0.24 \cdot (b_2 - b_3)}{0.392+0.24}\\
\end{matrix}
\end{equation}



 % praca moze byc podzielona na pliki

%\nocite{*} %wszystkie wpisy w bibliografi
\bibliographystyle{unsrt} %{latex8} %posortowane wzgledem wystepowania
\bibliography{bibliografia}

\addtocontents{toc}{\contentsline {chapter}{Bibliografia}{\thepage}{}}
%\listoftables
%\addtocontents{toc}{\contentsline {chapter}{Spis tabel}{\thepage}{}}
\listoffigures
\addtocontents{toc}{\contentsline {chapter}{Spis rysunków}{\thepage}{}}
%\lstlistoflistings
%\addtocontents{toc}{\contentsline {chapter}{Spis listingów}{\thepage}{}}
%\listofalgorithms % w zaleznosci od kompilatora i wersji klasy moga wystapic
%bledy przy kompilacji
%\addtocontents{toc}{\contentsline {chapter}{Spis algorytmów}{\thepage}{}}

%\biblioteka{tak} % tak/nie
\end{document}
